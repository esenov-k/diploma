\section{Выбор версии YOLO}

По итогам анализа методов детекции в предыдущей НИР, была выбрана нейросеть YOLO
за ее высокую скорость обработки видеопотока. 

YOLO или You Only Look Once — алгоритм детекции объектов, который обрабатывает
изображение только один раз. YOLO делит изображение на сетку, а затем использует
набор фильтров, чтобы предсказать все рамки и классификации за один раз. YOLO 
более эффективен для обнаружения больших объектов, но может пропустить маленькие
из-за фиксированного размера сетки, что не является критичным в рамках автономных
автомобилей. YOLO способна обрабатывать от десятков до сотен кадров в секунду.

Для выбора YOLO, проведем анализ существующих версий на основании результатов
сравнения семейств YOLOv5, YOLOv6, YOLOv7, YOLOv8 \cite{site::YoloCompare,
article::Hussain2023}. На рисунке \ref{fig::YoloCompare01} представлена диаграмма
сравнения средней точности по классам различных версий YOLO в зависимости от 
набора данных, которые обрабатывает нейронная сеть. 

\img[htb]{fig::YoloCompare01}{YOLO_comparison_01.png}{Диаграмма сравнения YOLO в
зависимости от категории}{0.8}

Как видно из диаграммы, 8-я версия YOLO показывает наиболее высокие результаты 
средней точности по классам (mean Average Precision (mAP)) -- дает общее 
представление о том, насколько хорошо модель работает по всем классам. 

На рисунке \ref{fig::YoloCompare02} представлено 2 графика, которые отражают 
зависимость mAP для разных семейств и версий YOLO от количества параметров и
задержки обработки одно изображения соответственно. 

\img[htb]{fig::YoloCompare02}{YOLO_comparison_02.png}{Графики сравнения YOLO в
зависимости от количества параметров и задержки обработки одно изображения}{0.8}

Как видно, 8я версия YOLO показывает наиболее высокую среднюю точность и наименьшую
задержку и, соответственно, наибольшую скорость, показывает YOLOv8n. Индекс n
означает, что данная модель принадлежить к семейству nano. На графиках также 
представлены small, medium, large, и extralarge. 

Таким образом, исходя из вышеприведенных диаграмм и графиков, можно сделать вывод,
что YOLOv8n подходит для наших целей по mAP и скорости обработки. 

\section {Выбор набора данных}

Датасет или набор тренировочных данных --- это важная часть обучения нейросети. 
Так как выбранная нейросеть будет использована в сфере автономного вождения, 
необходимо, чтобы разные виды ТС такие как легковые автомобили, грузовики, 
автобусы и т.д. были хорошо размечены в наборе данных. Также, наличие подробного
описания поможет в выборе нужных классов и данных для них, что в дальнейшем
сказывается на качестве обучения нейронной сети. Одним из важных факторов для 
набора тренировочных данных является вариативность содержащихся в нем данных.
Как было упомянуто выше, датасет должен содержать различные виды ТС, пешеходов и 
других возможных участников дорожного движения, причем, данные должны быть представлены
в различных условиях и не повторяться. Таким образом, составим ряд критериев, 
которые предъявим к нужному нам набору данных:

\begin{enumerate}

	\item Наличие подробного описания;
	
	\item Ориентированность на системы автономного вождения;
	
	\item Вариативность содержащихся данных.
	
\end{enumerate}

По выдвинутым критериям была выбран набор данных KITTI, который является одним из самых популярных
датасетов в сфере автономного вождения. Он имеет большое количество разнообразных данных, а также
хорошее описание. Еще одним плюсом можно выделить его популярность среди датасетов и наличие
широкой аудитории.


