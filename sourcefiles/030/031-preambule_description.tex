\section{Формальная постановка задачи}

Рассмотрим дискретные моменты времени $t_k$ с постоянным шагом $\Delta t$.
Все измерения переводятся в осевую систему координат \(\mathcal{V}\),
связанную с центром задней оси автомобиля; калибровочные матрицы
$\mathbf{T}_{\mathcal C \rightarrow \mathcal V}$ и
$\mathbf{T}_{\mathcal R \rightarrow \mathcal V}$ считаются известными.

Для каждого наблюдаемого объекта $j$ вводится вектор состояния
\[\mathbf x_k^{(j)} = \bigl[x,\;y,\;v,\;\theta\bigr]^{\mathsf T},\]
где $x$ и $y$~— координаты в плоскости дороги; $v$~— модуль скорости;
$\theta$~— азимут направления движения.
Динамика аппроксимируется моделью \\ \textit{Nearly‑CTRV}

\begin{equation}
\label{eq:nctrv}
\begin{aligned}
 x_k &= x_{k-1} + \frac{v_{k-1}}{\omega}\bigl[\sin(\theta_{k-1}+\omega\,\Delta t)-\sin\theta_{k-1}\bigr],\\
 y_k &= y_{k-1} + \frac{v_{k-1}}{\omega}\bigl[-\cos(\theta_{k-1}+\omega\,\Delta t)+\cos\theta_{k-1}\bigr],\\
 v_k &= v_{k-1} + a\,\Delta t,\\
 \theta_k &= \theta_{k-1}+\omega\,\Delta t,
\end{aligned}
\end{equation}

\noindent где $a$ и $\omega$~— продольное ускорение и угловая скорость,
а вектор процессного шума $\mathbf w_{k-1}$ распределён по
$\mathcal N(\mathbf 0,\,\mathbf Q)$.

Сенсоры формируют гибридный вектор наблюдения
\[\mathbf z_k^{(i)} = h\bigl(\mathbf x_k^{(j)}\bigr) + \mathbf v_k,\]
который объединяет прямоугольник детекции камеры
$[u,\,v,\,w,\,h]$ и радарные полярные величины
$[\rho,\,\varphi,\,\dot{\rho}]$;
шум измерений $\mathbf v_k$ подчиняется
$\mathcal N(\mathbf 0,\,\mathbf R)$.

В каждый момент времени доступно множество наблюдений
$Z_k=\{\mathbf z_k^{(1)},\dots,\mathbf z_k^{(m_k)}\}$.
Требуется определить матрицу ассоциации $\mathbf A_k$ между $Z_k$ и
совокупностью треков, оценить \emph{апостериорные}
$\{\hat{\mathbf x}_k^{(j)},\,\mathbf P_k^{(j)}\}$
и при этом удовлетворить ограничения:
\begin{itemize}
  \item системная латентность $\tau_{\text sys}\le 50\,\text{мс}$;
  \item средняя погрешность прогноза позиции (горизонт $0{,}5\,\text{c}$)
        $\operatorname{RMSE}\!\le\!0{,}20\,\text{м}$ при
        $v\le60\,\text{км/ч}$;
  \item показатели качества трекинга $\text{MOTA}\ge0.40$ и
        $\text{IDF1}\ge0.55$.
\end{itemize}

Задача решается схемой расширенного фильтра Калмана (EKF),
где стадия сопоставления реализована венгерским алгоритмом
с композитной стоимостью
\(c_{ij}=\alpha\,(1-\operatorname{IoU}_{ij})+\beta\,d_{\text{ReID}}\).

\section{Обоснование архитектурных решений}
\label{subsec:architecture_choice}

Анализ технического задания фиксирует четыре доминирующих ограничения:
точность трекинга, прогнозирование траектории, латентность~$\le50\,\text{мс}$
и ресурсный лимит платформы Jetson~Orin (40 Вт, 204 TOPS).
При сравнении доступных архитектур
(ByteTrack, DeepSORT, CenterTrack, GNN‑MOT и др.)
единственным вариантом, одновременно удовлетворяющим
всем ограничениям, оказалась связка
\textbf{YOLOv8n — DeepSORT — EKF}.

Главные аргументы приводятся ниже в повествовательной форме,
чтобы не дробить текст перечнями:

Во‑первых, YOLOv8 nano демонстрирует локально лучшую
комбинацию mAP$@0.5$\,=\,0.55 на наборе KITTI и времени инференса
\mbox{$\approx11$ мс} после компиляции TensorRT‑INT8;
этого достаточно, чтобы детектор занимал не более четверти
допустимого временного бюджета.
Во‑вторых, DeepSORT сочетает простоту реализации с
устойчивой идентификацией объектов благодаря
ReID‑эмбеддингу; косметическая замена 128‑мерного вектора на
40‑мерный INT8 снижает задержку ассоциации до \mbox{$\sim7$ мс}
без заметной деградации IDF1.
В‑третьих, линейный фильтр Калмана неприемлем при поворотах
пешеходов и велосипедистов: рефериные эксперименты
показали рост ошибки прогноза позиции до 0.35 м, тогда как
расширенный ФК даёт 0.18 м при дополнительной задержке всего 1.8 мс
на 25 активных треках.
Наконец, измерения радара естественным образом выражаются
в полярных координатах, а EM‑линейризация EKF позволяет
корректно объединить их с камерными детекциями без
опосредующей декартовой аппроксимации.

Поток данных включает раннее наложение радарных точек
на изображение и объектный уровень объединения в EKF;
сигнал наиболее важного объекта формируется по критерию
минимального времени до столкновения и~передаётся
в блок принятия решений ADAS.

Таким образом, выбранная архитектура обеспечивает:
\begin{enumerate}[label=\arabic*)]
  \item соблюдение требований по латентности и точности;
  \item масштабируемость при обновлении детектора или ReID‑сети;
  \item готовность к промышленной интеграции (пример — Polestar Pilot 2024).
\end{enumerate}
