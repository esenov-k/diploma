\section{Алгоритм DeepSORT}

При выборе алгоритмов слежения необходимо учитывать точность слежения за выделеными объектами, так как в плотном потоке, либо на перекрестке повышена вероятность ложных срабатываний, что может привести к ошибкам в управлении ТС.

DeepSORT (Simple Online and Realtime Tracking with Deep Association) — это усовершенствованный метод мультиобъектного трекинга, который сочетает алгоритм детекции объектов и алгоритм слежения, используя признаки объектов для повышения устойчивости трекинга~\cite{url:DeepSort, article:Bewley2016}. Основная цель DeepSORT — улучшение точности сопоставления объектов в последовательностях кадров видео. Алгоритм работает на основе фильтра Калмана для предсказания будущих состояний треков, а также использует алгоритм Венгера (Hungarian Algorithm)~\cite{book:Venger} для сопоставления треков и новых детекций. Кроме того, в DeepSORT интегрированы визуальные признаки, которые извлекаются с помощью глубокой нейронной сети (обычно ResNet). Эти признаки позволяют учитывать не только пространственное положение объектов, но и их визуальные характеристики.

Основные этапы работы алгоритма DeepSORT:

\begin{enumerate}
    \item Детекция объектов: на каждом кадре выполняется обнаружение объектов при помощи детектора (например, YOLO), который определяет ограничивающие рамки и категории объектов.
    \item Экстракция признаков: для каждого обнаруженного объекта вычисляются уникальные признаки, такие как пространственные характеристики и описания на основе сверточной нейронной сети.
    \item Ассоциация треков: сопоставление текущих детекций с существующими треками при помощи алгоритма "венгерского метода" (Hungarian Algorithm) и метрики расстояния Махаланобиса.
    \item Обновление треков: обновление положения треков и их состояния на основе фильтра Калмана.
    \item Инициализация и завершение треков: новые треки создаются для неассоциированных детекций, а треки завершаются, если они не обновлялись заданное количество кадров.
\end{enumerate}

\textbf{Математическое описание:}

Пусть $\mathbf{d}(i,j)$ — метрика расстояния между треком $i$ и детекцией $j$, вычисляемая как взвешенная сумма евклидова расстояния между признаками и расстояния Махаланобиса между позициями:
    \[ \mathbf{d}(i, j) = \alpha \mathbf{d}_{\text{appearance}}(i, j) + (1 - \alpha) \mathbf{d}_{\text{position}}(i, j) \]
где $\mathbf{d}_{\text{appearance}}(i, j)$ — расстояние между признаками объектов, $\mathbf{d}_{\text{position}}(i, j)$ — расстояние Махаланобиса между позициями.

\textbf{Ассоциация треков и детекций:}

Для сопоставления треков и детекций используется минимизация общей стоимости сопоставления с помощью венгерского алгоритма:
    \[ \min \sum_{i, j} c(i, j) x_{i,j} \]
где $c(i, j)$ — стоимость сопоставления пары трек-детекция, $x_{i,j}$ — бинарная переменная, равная 1, если трек $i$ сопоставлен с детекцией $j$, и 0 — иначе.

\textbf{Коррекция состояния:}

Текущее состояние трека обновляется по формуле:
    \[ \mathbf{x}_k = \mathbf{x}_k + \mathbf{K}_k (\mathbf{z}_k - \mathbf{H} \mathbf{x}_k) \]
    где $\mathbf{K}_k$ — матрица Калмана, $\mathbf{z}_k$ — вектор измерений, $\mathbf{H}$ — матрица измерений.

\textbf{Обновление ковариации ошибок:}
    \[ \mathbf{P}_k = (\mathbf{I} - \mathbf{K}_k \mathbf{H}) \mathbf{P}_k \]
где $\mathbf{P}_k$ — ковариционная матрица ошибок.


\textbf{Особенности применения:}
\begin{itemize}
    \item Высокая устойчивость к перекрытиям объектов за счёт использования признаков внешнего вида.
    \item Обновление состояния треков выполняется с помощью фильтра Калмана для повышения точности.
    \item Поддержка работы в реальном времени при использовании оптимизированных моделей детекции.
\end{itemize}

Обновление трека происходит, если они сопоставлены с детекциями, но если трек не обновляется в течение нескольких кадров, он удаляется. Новые детекции, не сопоставленные с существующими треками, инициируют новые треки.

Алгоритм DeepSORT обеспечивает точное и устойчивое слежение за объектами даже при сложных дорожных ситуациях, включая резкие перестроения и временные пропадания объектов из поля зрения, однако использования стандартного фильтра Калмана не учитывает нелинейности, которые необходимо учитывать в условиях сложных дорожных ситуациях и плотных потоках.

\subsection{Отличия и преимущества расширенного фильтра Калмана перед обычным фильтром Калмана}

Расширенный фильтр Калмана (EKF) является развитием стандартного фильтра Калмана (KF) и предназначен для работы с нелинейными системами. Основные отличия и преимущества EKF включают:

\textbf{1. Учет нелинейностей системы}

Обычный фильтр Калмана применим только для линейных систем, где динамика описывается линейными уравнениями. EKF расширяет область применения фильтра за счет использования нелинейных функций перехода состояния $f(\cdot)$ и измерений $h(\cdot)$. Это достигается путем линеаризации функций с помощью Якобианов на каждом временном шаге.

\textbf{2. Линеаризация и её влияние}

EKF аппроксимирует нелинейные функции в окрестности текущего состояния с помощью первых производных. Якобиан функции перехода состояния $f(\cdot)$ используется для предсказания ковариации, а Якобиан функции измерений $h(\cdot)$ — для корректировки состояния:

\begin{equation}
    \mathbf{F}_t = \frac{\partial f}{\partial \mathbf{s}} \big|_{\mathbf{s}_{t-1}}, \quad \mathbf{H}_t = \frac{\partial h}{\partial \mathbf{s}} \big|_{\mathbf{s}_t^-}.
\end{equation}

Эти матрицы позволяют учитывать нелинейное поведение системы, что улучшает точность предсказаний.

\textbf{3. Преимущества EKF}

\begin{itemize}

    \item EKF применим в задачах с нелинейной динамикой, таких как трекинг объектов с ускорением, изменение направления движения и работа с углами (например, ориентация робота).

    \item Обеспечивает более точное предсказание и обновление состояния для систем, где линейные модели недостаточны.

    \item Позволяет интегрировать сложные модели измерений, такие как данные от сенсоров, работающих с углами или расстояниями.
\end{itemize}

\textbf{4. Ограничения EKF}

\begin{itemize}

    \item Требует вычисления Якобианов, что увеличивает вычислительную сложность по сравнению с KF.

    \item Линеаризация вводит ошибки аппроксимации, особенно в системах с высокой степенью нелинейности.

    \item Зависимость от корректной математической модели системы: ошибки в определении $f(\cdot)$ и $h(\cdot)$ могут привести к значительным погрешностям.

\end{itemize}

\subsection{Методология интеграции EKF в DeepSORT}

Интеграция расширенного фильтра Калмана (EKF) в алгоритм DeepSORT требует замены стандартного фильтра Калмана на EKF, что улучшает возможности трекинга за счет учета нелинейностей в движении объектов. Методология интеграции включает следующие этапы~\cite{Wojke2017}:

\textbf{1. Замена линейного фильтра Калмана на EKF}

На этапе предсказания и обновления состояния стандартный линейный фильтр Калмана заменяется EKF. Это требует определения:

\begin{itemize}

    \item Нелинейной функции перехода состояния $f(\cdot)$, которая описывает динамику объекта.

    \item Нелинейной функции измерений $h(\cdot)$, связывающей состояние объекта с наблюдениями.

    \item Якобианов $\mathbf{F}_t$ и $\mathbf{H}_t$, необходимых для линеаризации функций $f(\cdot)$ и $h(\cdot)$.

\end{itemize}

\textbf{2. Расширение модуля предсказания}

Для предсказания будущего состояния и ковариации в DeepSORT используются нелинейные уравнения EKF:

\begin{equation}
\mathbf{s}_t^- = f(\mathbf{s}_{t-1}), \quad \mathbf{P}_t^- = \mathbf{F}_t \mathbf{P}_{t-1} \mathbf{F}_t^T + \mathbf{Q}.
\end{equation}

\textbf{3. Адаптация этапа обновления состояния}

Обновление состояния в EKF осуществляется с учетом измерений. В DeepSORT это реализуется через:

\begin{equation}
\mathbf{K}_t = \mathbf{P}_t^- \mathbf{H}_t^T (\mathbf{H}_t \mathbf{P}_t^- \mathbf{H}_t^T + \mathbf{R})^{-1},
\end{equation}

\begin{equation}
\mathbf{s}_t = \mathbf{s}_t^- + \mathbf{K}_t (\mathbf{z}_t - h(\mathbf{s}_t^-)), \quad \mathbf{P}_t = (\mathbf{I} - \mathbf{K}_t \mathbf{H}_t) \mathbf{P}_t^-.
\end{equation}

\textbf{4. Учет визуальных признаков}

DeepSORT использует визуальные признаки для улучшения сопоставления треков и детекций. EKF интегрируется таким образом, чтобы комбинировать предсказания положения объектов с результатами анализа визуальных признаков, сохраняя их значимость в процессе трекинга.

\textbf{5. Тестирование и валидация}

После интеграции EKF проводится тестирование алгоритма на наборах данных, таких как MOTChallenge, для оценки качества трекинга. Основные метрики включают точность ассоциации, количество ложных срабатываний и стабильность треков.
