\section{Интеллектуальные системы помощи водителю (ИСПВ)}

\textbf{Общая архитектура ИСПВ}

Рассмотрим общую архитектуру системы управления автономным транспортом, которая предствелена на рисунке~\ref{img:general_architecture}.

\img[!h]{img:general_architecture}{general_architecture.jpg}{Общая архитектура системы управления автономным транспортом}{0.6}

Данная архитектура может быть разделена на 3 уровня. Первый уровень -- уровень обработки данных, который обрабатывает данные с датчиков (радары, лидары камеры) и реализует детекцию и классификацию объектов, определяет их параметры, присваивает идентификационные номера и выделяет остальные параметры, необходимые для системы управления. 

Второй уровень -- уровень планирования, который включает в себя: планирование маршрута (глобальной траектории); модуль принятия решений, в котором такие системы, как АКК, АЭТ, СУП анализируют данные, пришедшие с первого уровня; планирование движения или постронение локальной траектории, для адаптации к конкретной ситуации на дороге. 

Третий уровень -- уровень управления. На данном этапе производится отработка локальной траектории путем генерации управляющих воздействий на тормозную систему, систему управления двигателем и систему управления рулевой рейкой.

\textbf{Мультиобъектное отслеживание}

Мультиобъектный отслеживание — это процесс идентификации и отслеживания нескольких объектов в пространстве и времени, при помощи компьютерного зрения. MOT важен для понимания и анализа сложных сцен, где множество объектов взаимодействуют друг с другом.

В ИСПВ МОТ играет ключевую роль, так как его задача обеспечивать устойчивое определение положения, скорости и идентичности объектов с течением времени. Эта информация критически необходима для таких ключевых функций ИСПВ, как автоматическое экстренное торможение (АЭТ), адаптивный круиз-контроль (АКК), система удержание в полосе (СУП) и т.д.

Мультиобъектное отслеживание выполняет функцию синхронизации и интеграции данных от различных сенсоров, обеспечивая временную и пространственную связность между измерениями. Это позволяет формировать целостную модель окружающей сцены, что, в свою очередь, служит основой для построения траекторий и выработки управляющих воздействий на исполнительные органы транспортного средства.

\textbf{Уровни автоматизации SAE}

В соответствии с SAE J3016 существует 6 уровней автоматизации автономных транспортных средств~\cite{Steckhan2022}:

\begin{itemize}

	\item 	0 уровень --- отсутствие автоматизации. На данном уровне управление автомобилем осуществляется исключительно водителем;

	\item 	1 уровень --- помощь водителю. В системах первого уровня часть функций управления автомобиля осуществляет автоматика, при этом водитель постоянно находится в готовности взять полное управление автомобилем на себя;
	
	\item	2 уровень --- частичная автоматизация. В таких системах автоматика полностью управляет автомобилем в ряде ситуаций, при этом водитель находится в готовности взять управление на себя в случае, если система не справляется;
	
	\item	3 уровень --- условная автоматизация. При определенных обстоятельствах система может сама выполнять все функции управления, но водитель все ещё должен быть готов взять на себя управление, когда система не способна выполнять свои функции;
	
	\item 	4 уровень --- высокая автоматизация. ТС может двигаться самостоятельно почти во всех ситуациях, но нестандартные дорожные ситуации или другие внешние факторы могут потребовать вмешательства человека, при этом человек может сидеть в пассажирском кресле;
	
	\item 	5 уровень --- полная автоматизация. ТС способно выполнять все функции управления в любых обстоятельствах без человеческого вмешательства.

\end{itemize}

МОТ может быть применим ко всем уровням автономности, кроме нулевого, в той или иной степени. Например, для уровней 1-2 MOT может использоваться для таких функций, как АКК и АЭТ, а уже для уровней с третьего и выше, MOT является критически важной составляющей, обеспечивающей осведомленность систем автомобиля о динамической дорожной обстановке.