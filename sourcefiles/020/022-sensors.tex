\section{Основные сенсоры, применяемые в МОТ}

Для повышения безопасности и эффективности движения, система помощи водителю должна иметь точную и подробную информацию об окружающей среде, которую она получает с датчиков. В зависимости от применяемого типа датчика, система получает определенную информацию о мире. В данной части будут рассмотрены основные сенсоры, которые применяются в ИСПВ, их преимущества и недостатки, а также возможность комбинирования друг с другом.

\textbf{Камеры}

Камера -- один из самых важных сенсоров для получения информации об окружающем мире, так как они формируют высокодетализированное цветовое изображение, пригодное для распознавания дорожной разметки, знаков и классификации объектов. Слабым местом камеры является чувствительность к погодным условиям и освещенности сцены.
 
\textbf{Радары}

Радар -- в рамках ИСПВ, данный сенсор позволяет выделять объекты вокруг ТС, измерять расстояние до них и скорость. Так как работа происходит в радиодиапазоне, радары практически не чувствительны к погодным условиям, однако имеют низкое угловое разрешение (градусы вместо угловых минут) и трудности с классификацией целей из-за слабого сигнального контраста.

\textbf{Лидары}

 Лидар -- один из самых универсальных датчиков, так как он предоставляет довольно 3-D геометрию окружающей среды и мало зависит от освещённости. Однако из-за высокой плотности сканирования, работа лидара может быть нарушена плохими погодными условиями. Также, цена этого датчика довольно высокая и его надежность во многом зависит от его конструкции. 
 
Для нивелирования недостатков каждого из сенсоров, используют сенсорное слияние (Sensor Fusion) [добавить источник]. Данная технология объединяет данные с различных датчиков, тем самым повышая точность и надежность данных, характеризующих окружающую среду. 

Комбинация камеры и радара позволит довольно точно распознавать дорожные знаки, разметку, объекты и при этом иметь информацию о расстоянии и скорости объектов вне зависимости от погодных условий. Такая комбинация является наиболее выгодной в плане стоимости и функциональности. 

Камера и лидар позволяют выстраивать 3D карту окружающей среды с цветовой и текстурной информацией для улучшения классификации. Однако восприимчивость к погодным условиям является серьезным недостатком. 

Радар и лидар дают возможность ИСПВ быть невосприимчивой к погодным условия и освещению, а также создавать 3D карту окружающей среды благодаря высокому угловому разрешению. Но при такой комбинации пропадает возможность определять разметку и дорожные знаки. 

Комбинация всех трех датчиков позволит использовать все доступные данные для наиболее полного восприятие окружающей среды. Несмотря на это, выбор такого типа сенсорного слияния требует больших вычислительных мощностей и сильно увеличивает стоимость системы в целом.

Таким образом, основываясь на опыте коллег программистов и инженеров, можно прийти к выводу, что использование камеры и радара покрывает практически весь спектр задач. Данная комбинация обеспечивает не только высокую точность определения расстояния и скорости объектов, но и позволяет эффективно распознавать все объекты, начиная от едущих рядом автомобилей, заканчивая дорожными знаками и разметкой. Относительно небольшая стоимость также является важным фактором выбора. 

В случае, если требования к точности и надежности очень высокие, возможна и интеграция лидара в систему сенсоров. 