\section{Работа с литературой}

Для хранения информации о источниках используется файл \textit{BibDB.bib},
заполненный специальным образом. При его написании следует руководствоваться
документацией к пакетам \lstinline:biblatex: и особенно
\lstinline:biblatex-gost:.  

Подключение библиографической базы к документу осуществляется командой
\lstinline:\bibliography{BibDB}:. 

Ссылка на источник в тексте дается с помощью команды \lstinline:\cite{label}:, а
вставка в документ списка литературы реализуется с помощью команды
\lstinline:\makebibliography:. Ее наличие в документе \textbf{обязательно} для
корректного создания \textbf{реферата}.

Стоит обратить внимание, что в файле \textit{BibDB.bib} источников может быть
больше, чем использовалось в документе, при этом будут напечатаны только те, на
которые была дана корректная ссылка. Если ссылки на литературу в документе
отсутствуют, то соответствующий раздел в итоговом файле представлен не будет.

Несколько ссылок на литературу: хорошая книга по \LaTeX~\cite{book:Lvov} и
стандарт, принятый обществом автомобильных инженеров Society of Automotive
Engineers (SAE) J3016~\cite{url:SAE_J3016}, декларирующий уровни автоматизации
транспортных средств (ТС). Кстати, на ссылки можно кликнуть.