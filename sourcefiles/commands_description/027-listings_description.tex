\section{Листинги}

В настоящем параграфе приведены примеры создания листингов. Для этого
предлагается использовать пакет \lstinline$listings$, название которого написано
как вставка кода в строку с использованием команды
\lstinline:\lstinline$listings$:. 

Вообще говоря, следует избегать вставки кода прямо в текст, так как это может
вызвать так называемые \lstinline: bad boxes : --- ситуации, когда \LaTeX{} не
может нормально распределить текст и он заезжает за пределы полей или
растягивается. Но в рамках настоящего документа эта рекомендация была опущена
(что вызвало порядка 15 таких ситуаций, которые были индивидуально решены ручным
способом).

Для вставки кода прямо в \textit{tex} файл можно воспользоваться окружением
\lstinline:lstlisting: с подачей необязательных параметров, как показано в блоке
кода ниже, а результат выполнения команд представлен в
листинге~\ref{code:hellow_world}.\\
\lstinline$\begin{lstlisting}[language=C++,$ \\
\lstinline$                       caption=Код програмы <<Hello World>>,$ \\
\lstinline$                       label=code:hellow_world]$ \\
\lstinline$#include <iostream>$ \\
\lstinline$ $ \\
\lstinline$int main() {$ \\
\lstinline$    std::cout << "Hello World!";$ \\
\lstinline$    return 0;$ \\
\lstinline$}$ \\
\lstinline$\end{lstlisting}$

\begin{lstlisting}[language=C++,caption=Код програмы <<Hello World>>,label=code:hellow_world]
#include <iostream>
	
int main() {
	std::cout << "Hello World!";
	return 0;
	}
\end{lstlisting}

Кроме того, листинг можно добавить из файла с помощью команды
\lstinline:\includelistingfile{path/to/file}{language}{caption}{label}:. 

\newpage
Листинг~\ref{code:example.py} сделан с помощью команды \\
\lstinline$\includelistingfile{listings/example.py}$ \\
\lstinline$                       {Python}$ \\
\lstinline$                       {Код программы <<example.py>>}$ \\
\lstinline$                       {code:example.py}$

\includelistingfile{listings/example.py}{Python}{Код программы <<example.py>>}{code:example.py}

При необходимости отключить цветной текст в листингах и во всем документе
(например, для успешной проверки программой TestVKR) необходимо в качестве
параметра класса указать \lstinline$monochrome$.