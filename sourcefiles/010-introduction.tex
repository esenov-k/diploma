\chapterwithoutnumber{Введение}

В наше время развитие робототехники происходит очень быстро. Затрагиваются практически все сферы нашей жизни, начиная от производств, заканчивая различными бытовыми делами. Роботы повсеместно внедряются и с каждым годом появляются новые системы и технологии, которые толкают процесс вперед.
 
Одним из наиболее актуальных и сложных направлений в робототехнике является роботизация транспортных средств. Создание полностью автономного ТС позволит повысить безопасность и эффективность дорожного движения. Такая система должна самостоятельно ориентироваться в пространстве, обнаруживать и классифицировать объекты вокруг, правильно оценивать дорожную ситуацию на дороге, принимать верные решения.

Для реализации всех вышеперечисленных задач инженеры разрабатывают сложные алгоритмы основанные на системах технического зрения, обнаружения, позиционирования, реализовывают сложнейшие системы управления, строят подробные математические модели ТС и окружающей среды для тестирования всех систем в совокупности. Тестирование всех систем может занимать от нескольких недель до нескольких месяцев, так как дальнейшее применение данных систем напрямую влияет на безопасность дорожного движения.

Одной из важнейших технологий в интеллектуальных системах помощи водителю является система мультиобъектного отслеживания (multi-object tracking, MOT), которая, как следует из названия, позволяет фиксировать, классифицировать и отслеживать объекты вокруг ТС. Применение MOT не ограничивается только наземными транспортными средствами. Беспилотные летательные аппараты, роботы доставщики, складские роботы и многие другие также полагаются на эффективное отслеживание объектов для выполнения поиска, слежения и остальных задач. 

Научная новизна данной работы заключается во внедрении  нелинейного алгоритма предсказания положения объекта при их частичном или полном перекрытии в условиях нагруженной дорожной ситуации. Такой подход позволит улучшить качество и точность отслеживания таких объектов как пешеходов, велосипедистов, мотоциклистов и других ТС, которые принимают участие в дорожном движении. 

