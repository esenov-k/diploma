\chapterwithoutnumber{Введение}

В наше время развитие робототехники происходит очень быстро. Затрагиваются практически все сферы нашей жизни, начиная от производств, заканчивая различными бытовыми делами. Роботы повсеместно внедряются и с каждым годом появляются новые системы и технологии, которые толкают процесс вперед.
 
Одним из наиболее актуальных и сложных направлений в робототехнике является роботизация транспортных средств. Создание полностью автономного ТС позволит повысить безопасность и эффективность дорожного движения. Такая система должна самостоятельно ориентироваться в пространстве, обнаруживать и классифицировать объекты вокруг, правильно оценивать дорожную ситуацию на дороге, принимать верные решения.

Для реализации всех вышеперечисленных задач инженеры разрабатывают сложные алгоритмы основанные на системах технического зрения, обнаружения, позиционирования, реализовывают сложнейшие системы управления, строят подробные математические модели ТС и окружающей среды для тестирования всех систем в совокупности. Тестирование всех систем может занимать от нескольких недель до нескольких месяцев, так как дальнейшее применение данных систем напрямую влияет на безопасность дорожного движения.

Одной из важнейших технологий в интеллектуальных системах помощи водителю является система мультиобъектного отслеживания (англ. multi-object tracking, MOT), которая, как следует из названия, позволяет фиксировать, классифицировать и отслеживать объекты вокруг ТС. Применение MOT не ограничивается только наземными транспортными средствами. Беспилотные летательные аппараты, роботы доставщики, складские роботы и многие другие также полагаются на эффективное отслеживание объектов для выполнения поиска, слежения и остальных задач. 

Актуальность данной работы обуславливается несколькими факторами. Один из главных факторов --- рост требований к безопасности. Функции, реализуемые в рамках ИСПВ — такие как автоматическое экстренное торможение (АЭТ), адаптивный круиз-контроль (АКК), предотвращение столкновений — невозможны без устойчивой работы модуля MOT, особенно в условиях плотного трафика. Также, говоря о плотном трафике, необходимо учитывать пешеходов, велосипедистов, мотоциклистов, которые демонстрируют сложные и часто нелинейные траектории движения, которые традиционным методам трекинга тяжело отслеживать, так как они, по-большей части основаны на линейных моделях прогнозирования. Это приводит к потере идентичности объекта (ID-switch), ошибкам прогнозирования и, как следствие, снижению надежности всей системы.
Еще одним фактором актуальности данной работы является переход к 3-4 уровню автономности (классификация SAE). Для функционирования на таких уровнях требуется высокая степень ситуационной осведомленности, включающая не только распознавание объектов, но и точный прогноз их будущего положения с учетом контекста и поведения окружающих агентов.

Научная новизна данной работы заключается во внедрении нелинейного алгоритма предсказания положения объекта при их частичном или полном перекрытии в условиях нагруженной дорожной ситуации. Такой подход позволяет повысить устойчивость и точность отслеживания пешеходов, велосипедистов, мотоциклистов и других транспортных средств даже в условиях частичной потери видимости объектов или их временного исчезновения из поля зрения.

Целью данной работы является разработка системы мультиобъектного отслеживания для роботизированного ТС на основе нелинейного алгоритма предсказания, обеспечивающая устойчивое предсказание положения объектов в условиях их частичного перекрытия и нагруженной дорожной ситуации. 

Задачи, решаемые в рамках ВКРМ:

\begin{enumerate}
    \item Провести обзор существующих методов детекции и трекинга объектов.
    \item Обосновать выбор модели YOLOv8n для детекции объектов.
    \item Реализовать модификацию DeepSORT с внедрением EKF.
    \item Разработать алгоритм определения наиболее важного объекта (НВО).
    \item Провести обучение и интеграцию модели, тестирование на датасетах.
    \item Сравнить с базовой реализацией DeepSORT + KF.
    \item Оценить возможности внедрения в ИСПВ уровня SAE 3–4.
\end{enumerate}

Объектом исследования в данной работе являются интеллектуальные системы помощи водителю (ИСПВ), а также алгоритмические и аппаратные компоненты автономных транспортных средств, обеспечивающие восприятие, интерпретацию и прогнозирование динамики объектов в окружающей среде.

Предметом исследования являются алгоритмы мультиобъектного отслеживания, реализованные на основе детекторов объектов и фильтров состояния, обеспечивающие устойчивое предсказание положения объектов в условиях высокой плотности дорожного трафика и частичных перекрытий.

В ходе выполнения ВКРМ были использованы следующие методы:

\begin{itemize}
    \item Алгоритмы фильтрации (Kalman Filter, EKF), методы ReID, DeepSORT.
    \item Фреймворки: YOLOv8 (Ultralytics), PyTorch, OpenCV.
    \item Языки программирования: Python, C++, MATLAB.
    \item Метрики оценки: MOTA, IDF1, FP, FN, IDSW.
    \item Оборудование: GPU RTX 3080, Linux, Python 3.10.
\end{itemize}

\textbf{База исследования и используемые источники.} Для обучения и тестирования использовались открытые датасеты KITTI и MOTChallenge, содержащие реальные сцены дорожного движения. Теоретическая часть основана на современных публикациях в области фильтрации состояний, визуального трекинга и нейросетевых детекторов. Разработка велась с использованием актуальных инструментов открытого программного обеспечения.

\textbf{Структура работы}:
\begin{itemize}
    \item Глава 1 — теоретический обзор ADAS, сенсоров и MOT;
    \item Глава 2 — архитектура системы, выбор моделей, модификация DeepSORT;
    \item Глава 3 — эксперименты, обучение, тестирование и сравнение;
    \item Глава 4 — анализ внедрения, ограничения и перспективы;
    \item Заключение — основные итоги и направления дальнейших работ.
\end{itemize}
